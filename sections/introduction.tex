\section{Introduction}

Wireless Sensor Networks (WSNs), has become an important branch of study for many application and simulation scenarios (internet of things, smart cities, medical systems and etc). Due to increased research in the WSN area, together with telemedicine, a new type of network emerged: the Wireless Body Area Networks (WBANs) or Body Area Networks (BANs). A WBAN consists of intelligent devices attached to the skin or implanted in the body capable to exchange data over a wireless network. With the advancement and improvement of WBANs, simulators have become an important tool that allows feasible tests with less cost and time. Castalia is a simulator of wireless sensor networks, body sensors and low power embedded devices. It is free and open source code, developed to users test their own algorithms and protocols \cite{b15}. 

In the WBAN simulation each sensor is represented by a generic node, which performs just transmission and reception of data. The objective of this work, therefore, is to propose a application layer of medical applications for the Castalia simulator, which will simulate a real behavior of a medical device (sensor or actuator), such as a thermometer, blood pressure monitor, glucose meter, and so on. These medical applications will be developed according to the standardization of communication of health devices defined by ISO/IEEE 11073 Personal Health Devices (X73-PHD). The X73-PHD standard describes the exchange, representation of data, and terminology for communication between Personal Health Devices (PHD).

The term PHD involves both medical devices and health and fitness devices used by "lay" users in their homes \cite{b3}. The ISO/IEEE 11073 family of standards is divided into three groups, the first and oldest part is the ISO/IEEE 11073 \textit{Lower Layer}, which specifies protocols and communication service using physical layers such as infrared, wireless RF or Ethernet \cite{b16}. The ISO/IEEE 11073 \textit{Point-of-Care-Devices} (X73-PoC) specifies communication standards for devices that are used exclusively in health facilities. Finally, the X73-PHD, sets standards for devices used by users in their homes.

The X73-PHD standard defines two types of devices: Agents and Managers. The agents are typically low power sensors or actuators, with limited processing power, whereas the managers are devices with a greater processing power, in which, they may or may not, be connected to a energy source. Although the 11073 standard does not explicitly comment, but device agents could be nodes of a WBAN. In this work we propose to make each node of a WBAN behave as an agent, that sends readings that a real agent would send.

\subsection{X73-PHD organization}

The IEEE Optimized Exchange
Protocol 11073-20601 is the core of X73-PHD family. It defines the communication syntax in the Domain Information Model (DIM), machine states and the services types in The Service Model and procedures in Communication Model. 
The Domain Information Model defines all common classes and data types used by the devices types. These classes are expanded by the specialization profiles according to the needs of each device. The Service Model defines the types of messages that can be exchanged between an agent and a manager and the conceptual context in which they are being transmitted \cite{b17}. The communication model defines the procedures to be followed under a normal operation, a exit condition, or when an error occurs.

The 11073 family of standards includes specialization profiles, that is, each agent has an associated standard that describes its data representation. For example, the standard 11073-10408 sets standards for a thermometer,
and the 11073-10415 for a balance. These specialization defines the DIM of each device, the attributes, methods, and events of each agent class.

In this work we used five different devices profiles: 11073-10406 Basic electrocardiograph (1 to 3 lead ECG), 11073-10404 Pulse oximeter, 11073-10408 Thermometer, 
11073-10417 Glucose meter, and 11073-10407 Blood pressure monitor. All theses agents was implemented in Castalia and can be used for WBAN simulations.

Antidote Stack is an implementation of the Optimized Exchange Protocol (IEEE 11073-20601) developed by Signove as part of the SigHealth Platform. This library is the first open source implementation of this standard. Was developed in ANSI-C with modular architecture, which allows code portability for different platforms. We adapted this library to work on Castalia.

The rest of the paper is organized as follows: In Section \ref{relatedworks} we survey related work on the area, focusing on X73-PHD works. In Section \ref{systemarch} we present an overall overview of the whole proposed system. In Section \ref{castaliaapplayer} we shows the parameters available for the user. Results are given in Section \ref{results} and finally, the conclusions in Section \ref{conclusion}.