\section{Related Works}

Wireless sensor network have been employed and developed to support the research on specific and general-purpose WSN. 
% **colocar na introdução ** Due to the high cost to deploy a real WSN system, simulators is a low-cost option to test the realistic behavior before network before the real implementation.
With the popularization of the 11073 standard several improvements and researches have been made for the smooth operation, interoperability between the devices, and for the simulators of WBAN's.

In \cite{b6} presents an open-source energy-harvesting simulation framework called GreenCastalia which supports multi-source and multi-storage energy harvesting architectures developed for the Castalia simulator. GreenCastalia project main focus is to simulate a realistic battery discharge on devices. The majors modifications to the original Castalia code was made to the Resource Manager module. Also in this module the authors adds the energy-harvesting systems that provides a more realistic battery model.  

In \cite{b7},\cite{b8},\cite{b9} and \cite{b10}  propose the integration of ISO/IEEE 11073 and the IOT protocols such as MQTT and COaP as transport protocols to enable the personal health devices directly shares health information through the Internet using low power consumption and few control messages. Also in those works is discussed the availability of enabling IOT technologies for the health information as well as the mapping of the messages from 11073 standard into the IOT protocols. Unlike the first work, these last four papers used real devices with Antidote as application layer protocol.

Another project on 11073 standards is \cite{b11} that has developed an interoperable  end-to-end remote patient monitoring platform using ZigBee Health Care Profile as transport layer and a Machine to Machine (M2M) solution to provide wide area network connectivity. This work also include a web application on the clinical side (server side) and use the stadandars and frameworks provided by Integrating the Healthcare Enterprise (IHE) \cite{b13} and Health Leven Seven (HL7) \cite{b12} to ensure end-to-end interoperability. These two companies advocate a world in which everyone can securely access and use the right health data when and where they need it.

A Point of Care version of the interoperability standard ISO/IEEE 11073 (X73-PoC) provide a mechanism to control remotely the agents. This mechanism is defined in the standards X73-10201 and X73-20301. However, the version of X73 oriented to Personal Health Devices has no mechanisms to do such a thing. So, in \cite{b14} the author proposes to adapt the remote control capabilities from X73-PoC to X73-PHD with an acceptable overhead and no extra cost to the manufacturers. This mechanism has to be installed in manager and in the agent. The remote control consists of being able to change units of measure from kilograms to pounds and vice versa directly from the manager, a smartphone or a compute engine
in nursing units.
